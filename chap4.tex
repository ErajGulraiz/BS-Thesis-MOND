\chapter{Summary and Conclusion}

Andromeda Galaxy is our nearest neighbour, a spiral galaxy just like the Milky Way, located at about 744 kpc. This thesis provides a technique to calculate astronomical data, and use it to reproduce the rotation curve of Andromeda Galaxy. From Section 3.1, we see that the graph only optimized for velocity shows disagreement with the observed curve, near the centre of the galaxy. When the same graph is optimized for mass, it is quite similar to the one obtained from observed data. We can conclude from this observation, that, a highly massive body is required at the centre of the galaxy to provide such a high velocity to the bodies in it. This also indicates the presence of a super massive black hole at the centre of the galaxy to provide enough gravitational force to retain such high velocities.
