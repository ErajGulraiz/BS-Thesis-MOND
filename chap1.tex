\chapter{Introduction}

Astronomy is the oldest of the natural sciences, the study of the celestial objects, their positions, and characteristics. Astrophysics allows us to understand Astronomy by applying the laws of Physics. In early times Astronomy only comprised of the observations of the motion of visible celestial objects, but with the passage of time, questions about the nature of the universe started to originate. Early studies only considered the Solar system but with the discovery of the telescope scientists realized that there is more to the universe. Now, in the era of modern Astronomy, scientists have found many astounding phenomenon happening in outer space \cite{introduction}.


\section{Literature Review}

\subsection{Evolution of the Universe}
 The origin and evolution of the Universe is explained by the widely accepted Big Bang Theory, according to which the universe was created as a consequence of a giant explosion of very dense and hot matter. After the explosion the universe started to expand and cool down; this cooling down resulted in the formation of elements and then in the formation of planets, stars, galaxies and various celestial objects \cite{evolution}.

\subsection{Missing Mass Problem}
When scientists studied the motion ofobjects in the Solar system they fell right into the realm of Newtonian laws of kinematics. There was no questioning these Laws until scientists studied large scale structures such as galaxies or cluster of galaxies. This was first observed by Jan Hendrick Oort in 1932, when he noticed that stars in the solar neighborhood moved faster than expected. 

In 1939, Horace Babcock measured the rotation curve of Andromeda Galaxy and concluded that the mass to luminosity ratio increases with radius \cite{missing_mass}. Similarly, in 1959 Louise Volders demonstrated that the spiral galaxy M33 does not spin according to Keplerian Dynamics \cite{mm}.

 The Rotation Curve gives us the information on how   velocity varies with distance from the centre of the system. There were two conclusions drawn from such discrepancies, that either our understanding of the Law Gravity is incorrect in the case of large scale structures, or there is some hidden matter that is providing enough force to these structures to move with observed accelerations.

\subsection{Dark Matter vs. MOND}
   In 1933 Fritz Zwicky invoked dark matter as a solution to this mass discrepancy problem by studying eight Coma galaxies. He calculated mass to light ratios of these galaxies and concluded that 90 percent of the mass that had been responsible for the observed ratio was missing \cite{dm_3}. 
   
   The other method to find out about missing mass was to study the rotational velocities of galaxies. It was assumed that the stars in the galaxies will follow  Kepler's Law just like the planets in the solar system, that is their velocities will show a decline as one moves away from the centre. So it was assumed that there is some invisible matter affecting their motion. That matter was named "Dark Matter".
   
\cite{mond} \cite{dm_1} \cite{dm_2} The mass discrepancy observed in the galaxies only occurs when the acceleration of gravity falls below a fix value that is $a_{0}= 1.2 \times 10^{-8} cm/s^2$ \cite{mond_1}. In 1983, Mordehai Milgrom proposed a modification to Newtonian Dynamics that resolves the missing mass problem without  the need for non-baryonic dark matter. It is also effective in describing the dynamics of galaxies and galaxy clusters and within some approximation, gravitational lensing \cite{mond_2} \cite{mond_3}. 

MOND and Dark Matter give conceptually different but operationally equivalent description of cosmic phenomenon.

\section{Aims and objective}
This research aims to reproduce the rotation curve of a specific galaxy by starting with observed luminosity data. We will compare the calculated rotation curve wit a direcctly observed rotation curve.

\section{Work Plan}
 This research is based on the work of Milgrom for solving the Missing Mass Problem. First, luminosity  data is collected using image analysis and then used to calculate the mass distribution of the galaxy. The mass distribution and the distance information is then used to calculate forcewhich is in turn used to calculate  the rotational velocity. Finally, the calculated  rotation curve is fitted to the observed rotation curve, using the parameters of central mass $ m_{0}$ and normalized peak velocity $ v_{0} $. 
