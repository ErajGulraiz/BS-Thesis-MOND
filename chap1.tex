\chapter{Introduction}

Astronomy is the oldest of the natural sciences, the study of the celestial objects,their positions, and characteristics. Astrophysics is the application of the Laws of Physics to understand Astronomy. In early times Astronomy only comprised of the observations of the motion of visible celestial objects, but with the passage of time, questions about the nature of the universe started to originate. Early studies only considered the Solar system but with the discovery of telescope scientists realized that there is more to the universe than just that. Now, in the era of modern Astronomy, scientists have found many astounding phenomenon happening in outer space \cite{introduction}.
The following research includes a study of the motion of a spiral galaxy namely Andromeda. The analysis of the motion will be done by studying the rotation curve of the specified galaxy. I will develop a set of calculations including computational work, that will prove us that the solution of the Missing Mass Problem can be found without invoking Dark Matter.
The first chapter of this thesis introduces the basic concepts that are the basis of the following research.
((written with progress))

\section{Literature View}

\subsection{Evolution of the Universe}
 The origin and evolution of the Universe is proposed by the widely accepted Big Bang Theory, according to which the universe was created as a consequence of giant explosion of very dense and hot matter. After the explosion it started to expand and cool down, this cooling down resulted in the formation of elements and then in the formation of planets, stars, galaxies and various celestial objects \cite{evolution}.

\subsection{Missing Mass Problem}
When the scientists studied the motion of the objects in the Solar system they fell right into the realms of Newtonian laws of kinematics. There was no question in these Laws until the scientists studied the large scale structures such as galaxies or cluster of galaxies. This was first observed by Jan Hendrick Oort in 1932, when he noticed that stars in the solar neighborhood moved faster than expected. In 1939, Horace Babcock measured the rotation curve to Andromeda Galaxy which resulted in the increasing mass to luminosity ratio \cite{missing_mass}. Similarly, in 1959 Louise Volders demonstrated that the spiral galaxy M33 does not spin according to the Keplerian Dynamics \cite{mm}. The Rotation Curves give us the information of change in velocity as we move away from the center of a system. There were two conclusions drawn from such errors, that either our understanding of the Gravity is not quite right, in the case when we study large scale structures or there is some hidden matter that is providing enough force to these structures to move with certain accelerations.

\subsection{Dark Matter vs. MOND}
   In 1933 Fritz Zwicky invoke the dark matter as a solution to this mass discrepancy problem by studying eight Coma galaxies. He calculated mass to light ratios of these galaxy and concluded that 90 percent of the mass that had been responsible for the observed ratio, was missing \cite{dm_3}. The other method to find about missing mass was to study the rotational velocities of galaxies. And it was assumed that the stars in the galaxies will follow the Kepler's law like the planets in the solar system, i.e. their velocities will show a decline as moving through the radius. So it was assumed that there is some invisible matter affecting their motion. That matter was named as Dark Matter.
\cite{mond} \cite{dm_1} \cite{dm_2} The mass discrepancy observed in the stellar systems only happens when the acceleration of gravity falls below a fix value i.e.$a_{0}= 1.2x10^-8 cm/s^2$.\cite{mond_1} In 1983, Mordehai Milgrom proposed a modification in the Newtons Dynamics that explains many properties of galaxies without  the need of non baryonic dark matter. It is also effective in describing the dynamics of galaxies and galaxy clusters and within some approximations, gravitational lensing \cite{mond_2} \cite{mond_3}. MOND and Dark Matter give conceptually different but operationally equivalent description of cosmic phenomenon.

\section{Aims and objective}

\section{Work Plan}
 This research is based on the work of Milgrom for solving the Missing Mass Problem. It focuses on calculation of Milgrom's constant to create an expected rotation curve without the mass descrepancy.
(still to work)
