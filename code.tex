\chapter{Code}
\section{Calculation of Distances and Intensity}
\begin{verbatim}
import matplotlib.pyplot as plt
import csv
import matplotlib.mlab as mlab
import numpy as np



import math
from scipy import misc
# np.seterr(over='ignore')


x0 = 88.23
y0 =1675.78

denom= math.sqrt(3)*255

def main():


    andro = misc.imread('andromeda.jpg')
    (w, h,d) = andro.shape



    with open('data.csv', 'w') as fout:

        for x in range(w):
            for y in range (h):

                r= calc_distance(x0,y0,x,y)
                i = calc_gs_intensity( andro[x,y] )
                m= i**0.25

                print (r,i,m)


                fout.write(str(r) + "," + str(i)+ "," + str(m) + "\n")





# i: numpy.ndarray with 3 values: R, G, B
# This function takes the array and converts from RGB to greyscale intensity value between 0 and 1
denom= math.sqrt(3)*255

def calc_gs_intensity(i):


    R = float(i[0])
    G = float(i[1])
    B = float(i[2])

    return math.sqrt(R*R+G*G+B*B)/denom


def calc_distance(x0,y0,x,y):

    x_dist = (x - x0)
    y_dist = (y-y0)

    return  math.sqrt(x_dist*x_dist+y_dist*y_dist)


if __name__ == '__main__':
    main()
\end{verbatim}

\section{Histogram for 400 Bins}
\begin{verbatim}
import csv
import matplotlib.pyplot as plt


def main():

    reader = csv.reader (open('data.csv', 'rb'), delimiter=',')

    R = []
    m= []

    for row in reader:
        r= float(row[0])
        #print r
        R.append( r* 0.005620)
        m.append( float(row[2]) )

    plt.hist(R, weights=m, bins = 400)
    n,bins,patches = plt.hist(R, weights=m, bins = 400) ## gets the value of bin boundaries and weights for each bin ..
    #print(n)
    #print (bins)

    with open('histdata.csv', 'w') as fout:

        for i in range(len(n)):
            fout.write(  str((bins[i] + bins[i+1])/ 2)+ ", " +str(n[i]) + "\n") ##calculates the midpoint of each bin and saves it to file. + saves the value of weights for each bin

    plt.title('Mass distribution of Andromeda')
    plt.xlabel('radius   (Kpc)    1pixel = 0.05620 kpc<  ')
    plt.ylabel('Masses relative to intensity')
    plt.show()




if __name__ == '__main__':

    main()
\end{verbatim}

\section{Observed Rotation Curve}
\begin{verbatim}
import numpy as np
import matplotlib.pyplot as plt
import pylab



x= []
y= []

with open ('rd.txt','r') as f:
    for row in f:

        x.append((row))



with open ('vrot.txt', 'r') as f1:
    for row1 in f1:

        y.append((row1))


pylab.ylim([0,400])
pylab.xlim([0,35])
plt.plot(x,y)
plt.scatter(x,y)


plt.title ('Andromeda Galaxy Rotation Curve')
plt.xlabel('distance from galactic centre in kpc')
plt.ylabel('rotational velocity km/s')
plt.show()
savefig('plot.jpg')

\end{verbatim}

\section{Force and Velocity Calculation and Optimization}
\begin{verbatim}
import csv
import numpy as np
import scipy
from scipy.special import ellipk, ellipe
import matplotlib.pyplot as plt
from pylab import *
import matplotlib.pyplot as plt
import pylab
from scipy.optimize import curve_fit
from numpy import array
import numpy

g = 6.67e-11
ms=1  #mass of the point
numpy.seterr('raise')

def main():

    Mass =[] #mass of the ring
    Rad=[]

    with open ('histdata.csv', 'r') as f:
        for row in f:
            values= row.split(',')

            Rad.append(float(values[0])) #saves values of radius from data file in empty list.
            Mass.append(float(values[1]))

    m0_o = Mass[0]

    # plt.plot(Rad,Mass)
    # plt.show()

    fit = Fit(Rad, Mass)

    x= []
    y= []

    with open ('rd.txt','r') as f:
        for row in f:

            x.append(float(row))



    with open ('vrot.txt', 'r') as f1:
        for row1 in f1:

            y.append(float(row1))



    R = numpy.array(x)
    V = numpy.array(y)

    #
    a,b = scipy.optimize.curve_fit(fit.v, R, V)
    print (a)
    print b


    v0 = a[0]
    m0 = a[1]
    v = fit.v(R,v0,m0)   #for black hole mass
    pmv = plt.scatter(R,v, color= 'g',label ='m0 & v0')
    plt.plot(R,v)


    vo = fit.v(R,v0,m0_o)  #for very small mass .. i.e the first entry in the list.
    pv = plt.scatter(R,vo , color='y', label = 'v0')
    plt.plot(R,vo)


    # a,b = scipy.optimize.curve_fit(fit.vn, R, V)   #optimizing without the mass.. only velocity will give a large value.
    # print (a,b)
    #
    # v0= a[0]
    #
    #
    # vm = fit.vn(R,v0)
    # plt.scatter(R,vm, color= 'c')
    # plt.plot(R,vm, color= 'c')

    # a,b = scipy.optimize.curve_fit(fit.K, R, V, None, None, False, True, ([1, 0, 1],[np.inf, np.inf, 1.5]))
    #
    # v0= a[0]
    # m0 = a[1]
    # km = a[2]
    #
    # kc = fit.K(R,v0,m0,km)
    # plt.scatter(R,kc, color= 'm')
    # plt.plot(R,kc, color= 'm')
    # print (v0,m0,km)

    # a,b = scipy.optimize.curve_fit(fit.v_mond, R, V)
    # print ('v_mond gives: \n')
    # print (a)
    # print b


    # v0 = a[0]
    # m0 = a[1]
    # a0 = a[2]
    # v_m = fit.v_mond(R, v0,m0,a0)
    # plt.scatter(R,v_m, color= 'c')
    # plt.plot(R,v_m, color= 'c')


    curve= velocity_curve(Rad,Mass)

    # po= act_curve(x,y)
    # plt.legend([po,pv,pmv ], ['Observed', 'Optimized: v0','Optimized: v0 & m0 ' ])
    plt.show()

def act_curve(x,y):

    # x = x[:100]
    # y = y[:100]


    # pylab.ylim([0,400])
    # pylab.xlim([0,35])
    plt.plot(x,y, label = 'Observed')
    po= plt.scatter(x,y, color ='r')



    plt.title ('Andromeda Galaxy Rotation Curve')
    plt.xlabel('distance from galactic centre in kpc ')
    plt.ylabel('rotational velocity km/s')
    # plt.show()
    return po



def velocity_curve(Rad, Mass):

    force = []
    vel = []
    j=200

    for r in Rad:       ##for force inside the point we want to calculate, r will be fixed and R will be changing.
        T=0

        for i in range(len(Rad)):
            R= Rad[i]
            M= Mass[i] if i == j else 0
            if  r!=R:
                lF = F(ms,M,R,r)
                # print(lF,r,R,M)
                T= T+lF

        force.append(T)

        print(r,T)
    # print(len(force))
    # print (len(Rad))
    print (F(ms,10,13,0.001))


    plt.plot(Rad,force, color = 'm')
    plt.title('Force vs.radius (for a single ring)')
    plt.xlabel('Radius (in kpc)')
    plt.ylabel('Force')


    # for j in range(len(Rad)):
    #     r=Rad[j]
    #     f= force[j]
    #     v= ((-f*r)/ms)**0.5
    #     # print (r,f,v)
    #     vel.append(v)
    # plt.plot(Rad,vel)
    # plt.scatter(Rad,vel, color = 'c')
    # plt.title('Velocity vs Radius')
    # plt.xlabel('Radius (in parsecs)')
    # plt.ylabel('velocity')

    plt.show()

def q(r,R) :

    return r*r+R*R+2*r*R

def k(r,R,con) :
    b = float(4*r*R)
    return (b/con)

def f(m,M,R,r,con,ks,g):
    d= 2*R - ks * (r+R)
    E=  ellipe(ks**0.5 )##(elliptic function of 2nd kind)
    j= (1-ks)
    G= 2*R* ellipk(ks**0.5) ## elliptic function of 1st kind
    h= math.pi * (con ** 1.5) *j

    return (((-1*M*ms*g)/ h) * (1/ks)* (G *j -E *d) )

def F(ms,M,R,r):

    # print(r)s
    # print(R)
    con = q(r,R)
    # print (con)
    ks = k(r,R,con)
    # print (ks)
    g = 6.67e-11

    # fa = f(m,M,R,r,con,ks,g)
    # print (fa)


    return f(ms,M,R,r,con,ks,g)


class Fit:

    def __init__(self, Rad, Mass):

        self.Rad = Rad
        self.Mass = Mass


    def v(self, rs, v0, m0):

        vs=[]

        self.Mass[0] = m0

        for r in rs:

            T=0

            for i in range(len(self.Rad)):
                R= self.Rad[i]
                M= self.Mass[i]
                if abs(R-r) > 0.5:
                    lF = F(ms,M,R,r)
                    T= T+lF

            v = v0 * (abs(T*r))**0.5
            vs.append(v)

        return vs

    def vn(self, rs, v0):

        vs=[]

        for r in rs:

            T=0

            for i in range(len(self.Rad)):
                R= self.Rad[i]
                M= self.Mass[i]
                if abs(R-r) > 0.5:
                    lF = F(ms,M,R,r)
                    T= T+lF

            v = v0 * (abs(T*r))**0.5
            vs.append(v)

        return vs

    def K(self, rs, v0,m0, km):  #any optimizing constant ... now not required.

        vs=[]
        self.Mass[0] = m0

        for r in rs:

            T=0

            for i in range(len(self.Rad)):
                R= self.Rad[i]
                M= self.Mass[i]*km
                if abs(R-r) > 0.5:
                    lF = F(ms,M,R,r)
                    T= T+lF

            v = v0 * (abs(T*r))**0.5
            vs.append(v)

        return vs

    def v_mond(self, rs, v0,m0,a0):  #fitting for mond .. which didnt work..

        vs=[]

        for r in rs:

            T=0

            for i in range(len(self.Rad)):
                R= self.Rad[i]
                M= self.Mass[i]
                if abs(R-r) > 0.5:
                    lF = F(ms,M,R,r)
                    T= T+lF


            v = v0 * ((a0/((ms*a0)-T))*abs(T*r))**0.5
            vs.append(v)

        return vs


if __name__ == '__main__':
    main()




\end{verbatim}